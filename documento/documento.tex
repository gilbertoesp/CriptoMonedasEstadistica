\documentclass[12pt,letterpaper]{article}
\usepackage[T1]{fontenc}
\usepackage[utf8]{inputenc}
\usepackage{amsmath}
\usepackage{amsfonts}
\usepackage{amssymb}
\usepackage{graphicx}

%\usepackage[spanish]{babel}
\graphicspath{ {img/} }

\usepackage[left=2cm,right=2cm,top=2cm,bottom=2cm]{geometry}
\title{Criptomonedas}
\author{Gilberto Espinoza, Luis Fernando Sotomayor}
\begin{document}
\maketitle
\abstractname{\\Criptomonedas, son dinero digital o virutal, que se basan en la encriptaci\'on para su seguridad. Estas nuevas formas de realizar transacciones estan revolucionando el mercado, dado que en principios de este a\~no, 2017, la lider en esta nueva tecnologia, \textbf{Bitcoin}. Este ten\'ia un precio de \$933.66 USD y para principios de noviembre del mismo a\~no alcanzaba \$6440.97 USD, un crecimiento del mas 600\% en unos cuantos meses. Marcando una nueva moda nuevas monedas empezaron a emerger, el como estas estan variando, es el objetivo de este documento. }

\section*{Criptomonedas}
	Para explicar las criptomonedas utilizaremos al Bitcoin como referencia, dado que fue esta la que empezo el moviemiento y puede decirse que cualquier otra es una modificacion o copia directa.

	Una versi\'on puramente electr\'onica de efectivo permitir\'ia que los pagos en l\'inea fuesen enviados directamente de un ente a otro sin tener que pasar por medio de una instituci\'on financiera. Las transacciones son validadas por una red de usuarios, que verificaran que ocurra un doble gasto. Cuando un usuario malintencionado intenta gastar sus criptomonedas en dos destinatarios al mismo tiempo se denomina doble gasto. La miner\'ia y la cadena de bloques permiten crear un consenso en la red acerca de cu\'al de las dos transacciones es considerada v\'alida.
	
	Una criptomoneda puede utilizarse como cualquier divisa, intercambiarla por alguna otra o gastarla por algun servicio o producto que la acepte.
	\subsection*{Mision y vision}
	La misi\'on de esta nueva tecnolog\'ia es dar alternativas a las personas de las monedas reguladas y controladas por alg\'un gobierno entonces el usuario pueda experimentar una condici\'on financiera mas libre y directa. Esto se quiere lograr independientemente de la condici\'on financiera del usuario ya que es un alto porcentaje no cuenta con una cuenta de banco com\'un.
	
	La visi\'on es lograr que las criptomonedas sea mundialmenten aceptadas por los negocios, empresas y los ciudadanos del d\'ia al d\'ia para que estos operen sin necesidad de terceros, ya sean bancos, gobiernos o parecidos. 	
	
	El objetivo con las criptomonedas se guarda en que estas deben ser descentralizadas, no imparta que tanto dinero o poder tenga una persona o institucion no puede adue\~uarse o monopolizar la moneda, asi como el gobierno de Estados Unidos y sus bancos son los que manejan los dolares (USD), en las criptomonedas esto no es posible gracias a la tecnologia P2P. 
	Las criptomonedas ofrecen transacciones internacionales sin impuestos o tarifas ya que son directas entre los usuarios, se validan entre la comunidad, no por un organismo en el que ambos confien, un ejemplo de este tercero seria Western Union, empresa cuyo servicio es la transaccion de dinero en efectivo a nivel mundial y la cual cobra tarifas.
	\subsection*{Como funcionan}
	
		\subsubsection*{Blockchain}
		La cadena de bloques es un registro p\'ublico de las transacciones Bitcoin en orden cronol\'ogico. La cadena de bloques se comparte entre todos los usuarios de Bitcoin. Se utiliza para verificar la estabilidad de las transacciones Bitcoin y para prevenir el doble gasto. Es una contabilidad p\'ublica compartida en la que se basa toda la red de la criptomoneda. Todas las transacciones confirmadas se incluyen en la cadena de bloques. De esta manera los monederos Bitcoin pueden calcular su saldo gastable y las nuevas transacciones pueden ser verificadas, asegurando que el cobro se esta haciendo al que realiza el pago. La integridad y el orden cronol\'ogico de la cadena de bloques se hacen cumplir con criptograf\'ia.
		
		Algunas criptomonedas al crear un nuevo bloque genera nuevas monedas y se las transfiere a la persona/minero/cliente que verific\'o y envi\'o el bloque al blockchain.

Por ejemplo Bitcoin, cada nuevo bloque minado genera una recompensa en Bitcoins que se entregan al minero que gener\'o el bloque.

Una transacci\'on es una transferencia de valores entre monederos Bitcoin que ser\'a incluida en la cadena de bloques. Los monederos Bitcoin disponen de un fragmento secreto llamado clave privada, utilizada para firmar las operaciones, proporcionando una prueba matem\'atica de que la transacci\'on est\'a hecha por el propietario del monedero. La firma tambi\'en evita que la transacci\'on no sea alterada por alguien una vez \'esta ha sido emitida. Todas las transacciones son difundidas entre los usuarios y por lo general empiezan a ser confirmadas por la red en los 10 minutos siguientes a trav\'es de un proceso llamado miner\'ia.

		\subsubsection*{Mineros}
		La miner\'ia es un sistema de consenso distribuido que se utiliza para confirmar las transacciones pendientes a ser incluidas en la cadena de bloques. Hace cumplir un orden cronol\'ogico en la cadena de bloques, protege la neutralidad de la red y permite un acuerde entre todos los equipos sobre el estado del sistema. 
		Estas normas impiden que cualquier bloque anterior se modifique, ya que hacerlo invalidar\'ia todos los bloques siguientes. La miner\'ia tambi\'en crea el equivalente a una loter\'ia competitiva que impide que cualquier persona pueda f\'acilmente añadir nuevos bloques consecutivamente en la cadena de bloques. 
		De esta manera, ninguna persona puede controlar lo que est\'a incluido en la cadena de bloques o reemplazar partes de la cadena de bloques para revertir sus propios gastos.
		
Usar potencia de procesamiento para producir un bloque v\'alido, y como resultado "extraer" algunas bitcoins. Las reglas de la red se establecen de forma que la dificultad se ajusta para mantener la producci\'on de bloques en aproximadamente uno cada 10 minutos. De esta forma, un mayor n\'umero de mineros participantes en la actividad de miner\'ia, implicar\'a una mayor dificultad en la generaci\'on de un bloque para cada minero individual. Una mayor dificultad total implicar\'a, para un atacante, una m\'as dif\'icil sobreescritura del extremo de la cadena de bloques con sus propios bloques (lo que le permitir\'ia el doble gasto de sus monedas.

Consiste en un tipo de trabajo que realiza un cliente, que por lo general es la realizaci\'on de un c\'omputo en un ordenador, ese trabajo es verificado en el servidor. Lo com\'un es que estos c\'omputos deben ser dif\'iciles para el cliente pero debe ser f\'acil de verificar por el servidor. En Bitcoin se usa POW para verificar transacciones y generar nuevos bloques, este proceso se conoce como minado (mining).

Adem\'as de la importancia para el mantenimiento de la base de datos de transacciones, la miner\'ia es tambi\'en el mecanismo por el que las bitcoins son creadas y distribuidas a las personas en la econom\'ia bitcoin. Las reglas de la red se establecen de tal forma que en los pr\'oximos cien años, d\'ecadas m\'as o menos, ser\'an creadas un total de 21 
millones de bitcoins. 

	Tambi\'en existe la opci\'on de comprar poder de computo online a alguna empresa, esto es pagar una cantidad de dinero para que un hardware trabaje minando las monedas sin que uno mismo tenga que conseguir el equipo necesario, esta una opci\'on pero representa un riesgo ya que tu mismo no controlas el hardware.
	
	Una opci\'on que esta siendo muy popular ultimamente es es el "Pool Mining", un grupo de mineros, se dedican a resolver un proceso y como comparten recursos, reparten la recompesa pero esta la consiguen mas seguido al tener mas podera su disposici\'on.
	
		\subsubsection*{Centros cambio}
En esta carrera por alcanzar la c\'uspide de la innovaci\'on lideran pa\'ises como Argentina, Brasil y M\'exico. Esta \'ultima naci\'on se ha destacado por innumerables proyectos de desarrollo de la tecnolog\'ia de contabilidad distribuida (DLT) y un ecosistema cada vez m\'as nutrido y variado de startups, que le depara un prometedor futuro tecnol\'ogico a la econom\'ia mexicana.		

En conjunto con el sector empresarial dedicado a las tecnolog\'ias de contabilidad distribuida, las monedas criptogr\'aficas han llegado a M\'exico con el objetivo de incluir a la poblaci\'on a un sistema financiero m\'as accesible y justo; ofreci\'endole tanto facilidades transaccionales a los ciudadanos como la posibilidad de invertir a largo plazo.

\textbf{Bitso}

M\'exico alcanz\'o durante este año su m\'aximo hist\'orico de transacciones bitcoin con 242 mil d\'olares en transacciones locales, cifras que aumentaron en conjunto con otros pa\'ises de Latino\'america. Asimismo, esta criptomoneda tiene un fuerte ecosistema en el pa\'is con m\'as de 81.000 usuarios, seg\'un datos de una de las mayores casas de cambio del pa\'is, Bitso.

\textbf{Volabit}

Volabit es otra casa de cambio mexicana que ofrece compra de bitcoins por medio de transferencias o dep\'ositos de dinero fiat, permitiendo tanto a las personas que poseen cuenta bancaria como a los desbancarizados acceder al dinero criptogr\'afico. Los usuarios pueden depositar sus pesos en los locales 7-Eleven, las farmacias Benavides, farmacias del Ahorro o Extra
\subsubsection*{Cajeros Automaticos}
Otro m\'etodo para acceder a unos cuantos bitcoins son los cajeros autom\'aticos especializados en esta criptomoneda. M\'exico es el segundo pa\'is de Am\'erica Latina con mayor cantidad de ATM de bitcoin, contando con tres (3) dispositivos instalados en todo su territorio —seg\'un fuentes de Coin ATM Radar—; la naci\'on es superada solamente por Rep\'ublica Dominicana que posee cuatro (4) cajeros en su capital, marca que se rompi\'o recientemente.

Los pobladores y turistas de las ciudades de Tijuana, La Fonda y Ciudad de M\'exico han sido los beneficiados de estos servicios. En el caso de Tijuana, el ATM se encuentra en el local IMAXESS – Diagnostic Imaging, espec\'ificamente en el sector Madero, calle Jalisco. La m\'aquina permite tanto comprar bitcoins con dinero fiat, como retirar fondos en monedas criptogr\'aficas a cambio de pesos mexicanos; con un l\'imite de compras de $750,000$ pesos y un m\'aximo de venta de $74,500$ pesos mexicanos.

Por otro lado, en Ciudad de M\'exico el cajero se ubica en el local Fantastico Comics del sector Felix Cuevas. El dispositivo permite no s\'olo comprar bitcoins, sino tambi\'en litecoin y dash. Asimismo, el ATM compra y vende criptomonedas bajo un m\'aximo de 6,000 pesos mexicanos.

Por \'ultimo, el hotel y restaurante La Fonda en la carretera Tijuana Ensenada, tambi\'en cuenta con su propio cajero que \'unicamente acepta bitcoins, el cual solo permite comprar esta criptomoneda por un m\'aximo de $20,000$ pesos mexicanos.
	\subsection*{Bitcoin}
        \subsubsection*{Historia}
Bitcoin es la primera implementaci\'on de un concepto conocido como "moneda criptogr\'afica", la cual fue descrita por primera vez en 1998 por Wei Dai en la lista de correo elect\'onico "cypherpunks", donde propuso la idea de un nuevo tipo de dinero que utilizara la criptograf\'ia para controlar su creaci\'on y las transacciones, en lugar de que lo hiciera una autoridad centralizada.

 La primera especificaci\'on del protocolo Bitcoin y la prueba del concepto la public\'o Satoshi Nakamoto en el 2009 en una lista de correo electr\'onico. Satoshi abandon\'o el proyecto a finales de 2010 sin revelar mucho sobre su persona. Desde entonces, la comunidad ha crecido de forma exponencial y cuenta con numerosos desarrolladores que trabajan en el protocolo Bitcoin.

La anonimidad de Satoshi a veces ha levantado sospechas injustificadas, muchas de ellas causadas por la falta de comprensi\'on sobre el c\'odigo abierto en el que se basa Bitcoin. El protocolo Bitcoin y su software se publican abiertamente y cualquier programador en cualquier lugar del mundo puede revisarlo o crear su propia versi\'on modificada del software. 

Al igual que los programadores actuales, la influencia de Satoshi se ha limitado a que los cambios que hizo los adoptaran los dem\'as y, por tanto, no controlaba Bitcoin. As\'i, conocer la identidad del inventor del Bitcoin es igual de relevante que saber qui\'en invent\'o el papel.
            
                \subsubsection*{Satoshi Nakamoto}
Satoshi Nakamoto es la persona o grupo de personas que crearon el protocolo Bitcoin y su software de referencia, Bitcoin Core. 
En 2008, Nakamoto public\'o un articulo acerca de Bitcoin en el sitio de criptografia $mewdowd.com$. 

En 2009, lanz\'o el software Bitcoin, creando la red del mismo nombre y las primeras unidades de moneda.

Se desconocen su identidad y su nacionalidad. Si bien los pocos datos disponibles sobre \'el apuntar\'ian a Jap\'on, nunca escribi\'o absolutamente nada en japon\'es, ni hizo versi\'on japonesa del cliente Bitcoin ni una p\'agina inicial en japon\'es para bitcoin.org.

Por todo lo que se sabe, es totalmente desconocido fuera de Bitcoin, y su clave PGP se cre\'o apenas unos meses antes de la fecha del bloque de g\'enesis. Parece estar muy familiarizado con la lista de correo sobre criptograf\'ia, pero en esa lista no hay mensajes suyos no relacionados con Bitcoin. Utilizaba una direcci\'on de correo electr\'onico de un servicio an\'onimo de alojamiento de correo (vistomail) as\'i como otra de una cuenta gratuita de correo web (gmx.com) y enviaba siempre los mensajes a trav\'es de una conexi\'on Tor. Se ha especulado con que su identidad habr\'ia sido creada expresamente con antelaci\'on como una manera de protegerse a s\'i mismo o a la red Bitcoin. Es posible que eligiera el nombre "Satoshi" porque puede significar sabidur\'ia o raz\'on. 

            \subsection*{Eventos sociales, politicos y economicos que impactaron el Bitcoin}
           	\subsubsection*{Corea de Norte ataca casas de cambio}
           	Sin embargo, siempre hay una parte negativa para cada situaci\'on, y una de ellas ha sido el aumento de acciones il\'icitas para poder tener un posesi\'on al menos un bitcoin. Una de las trabas m\'as renombradas en los \'ultimos ha sido Kim Joung-un y Corea del Norte, pues seg\'un expertos en seguridad los hackers del pa\'is est\'an buscando la manera de acceder a casas de cambio de bitcoin.

Corea del Norte tiende a enfocar su espionaje cibern\'etico, en actividades relacionadas con el Estado. Sin embargo esto cambi\'o desde el año pasado, cuando la compañ\'ia de ciberseguridad Fireeye empez\'o que darse cuenta que el pa\'is estaba poniendo como blanco a entidades bancarias y a todo el sistema financiero mundial. En 2017 se han detectado diversos ataques a las casas de cambio de Corea del sur, y esta actividad ya se est\'a expandiendo a grupos bancarios en Europa, incluyendo a una compañ\'ia de cajeros autom\'aticos.
               
               Si bien se sabe que Corea del Norte siempre est\'a detr\'as de alguna estrategia sospechosa, estos ataques constantes a las casas de cambio de Bitcoin muestran un nivel considerable de desesperaci\'on. El pa\'is se encuentra aislado del mundo debido a sanciones a nivel mundial, y las mismas solo cobraron m\'as poder desde que Donald Trump se instal\'o en la Casa Blanca.
              

        \subsection*{Usuario final}

            \subsection*{Wallets}
Un software que se comunica con la red para poder realizar operaciones de env\'io y recepci\'on de la criptomoneda.
                \subsubsection*{Que son}
                Las \textit{wallets} guardan las llaves privadas que tu necesitas para acceder a la dirreccion del bitcoin y gastar tus fondos. Vienen de distintas formas, dise\~nadas para distintos dispositivos.
                
                La wallet oficial de Bitcoin es \textit{Bitcoin Core} la misma palicacion se utiliza para la mineria si es que sea desea, pero hay muchas alternativas, una para cada nueva moneda que emerge, adicionalmente hay wallets que soportan diferentes monedas e incluso puedes hacer el cambio de una a otra desde la misma. Estas crecen cada vez, dandote en tiempo real varaiciones de los precios.
                \subsubsection*{Como funcionan}
				Estas guardan tu "dirrecc\'on", tu llave privada \'unica para que se use de referencia para que los demas usuarios puedan mandarte monedas, tambien, pueden generar un codigo QR, que al ser escaneado ejecuta la transferencia, entonces con simplemente compartir esta imagen uno puede mandar y recibir monedas digitales, por lo cual esta informacion debe ser tratada con sumo cuidado                                
            \subsubsection*{Como obtener Bitcoins}
	Uno puede obtenerlos de diversas maneras, casas de cambio, trabajo por ello, o una simple transferencia en un local. La forma mas com\'un es comprarlos a trave\'es de una casa de cambio, en M\'exico podemos mencionar \textit{Bitso} y \textit{Volabit} empresas que estan respetando las regulaciones que rodean las criptomopnedas.
	
	Otro m\'etodo es el tipico minado pero para un usuario que no quiere o no pueda sumergirse mucho en el mundo de esta nueva tecnologia, llega a ser un metodo que involucra mucha energia y tiempo.
            \subsubsection*{Empresas que aceptan bitcoin como metodo de pago}
\textbf{Dish}
La cadena de television de paga empezo a aceptar Bitcoin desde Agosto del 2014. Esta esta asociada a \textit{Coinbase.com} para realizar tales trasacciones.

\textbf{Newegg}
La tienda en linea para la compra de gadgets, partes y equipos de computo no pudo quedarse atr\'as y empez\'o a recibir bitcoin como metodo de pago utilizando a BitPay como medio para el manejo de cambio.

\textbf{Steam}
La plataforma de compra de videojuegos online acepto Bitcoin como metodo de pago cuando la moneda apenas valia \$20 USD.
\section*{Criptomonedas vs Moneda clasica}

	\subsection*{Diferencias}
	EL dinero clasico puede imprimirse y reproducirse tanto como el gobierno lo desee provocando inflacion y que el poder adquisitvo disminuya, BTC no puede ser reproducido si no creado, recordemos que el el limite es 21 millones de unidades.
	
	Descentralizacion, BTC no depende de alguna institucion para tener valor o para utilzarse, se basa en si misma, confia en si misma y los usuarios comparten esa confianza, aceptamos dinero clasico por que los demas lo aceptan, pero este al final depende de las accciones de los bancos, la bolsa y el gobierno; BTC depende unicamente de sus usuarios.
	
	Primero que todo Bitcoin no pertenece a ning\'un Estado o pa\'is y puede usarse en todo el mundo con igualdad. Esto lo hace de mucha utilidad, en especial para las personas que no poseen cuentas bancarias tradicionales. El Bitcoin se puede cambiar a euros u otras divisas y viceversa, como cualquier moneda. 
	
	F\'acil proceso de creaci\'on de su cuenta digital Bitcoin. Ciertamente, es m\'as f\'acil crear una cuenta digital de Bitcoin que crear una cuenta bancaria tradicional. Cualquier persona puede crear una cuenta digital en segundos, sin tener que proveer sus detalles personales, y sin enviar su historial crediticio. Adem\'as, la rata de aceptaci\'on de la cuenta digital Bitcoin es del 100%.
	
	\subsection*{Quien "controla" cambio de Bitcoin a USD}
	Nadie, los intercambios entre divisas tipicas a BTC, son realizadas por empresas pero estas no lo controlan, solo ofrecen el servicio a una escala mas grande. Comprar criptomonedas de algun centro de cambio no seria distinto a comprarle a tu amigo unos y pagarle en efectivo.
		\subsubsection*{Como funciona este cambio}
        Se realiza de igual manera que comprar algun otro servicio con su tarjeta bancaria, pagamos la luz, el telefono el cable y se descuenta el total entonces tienes tu servicio en tu casa, con bitcoin enlazas tu "wallet" es decir, un direccion \'unica que te identifica como usuario de bitcoin, esta tu la guardas y proteges ya que esta es como una cuenta donde los datos de tus bitcoin son guardados y ese es el "servicio" que compraste
% Redundante respecto a otras secciones como diferencias
%    \subsection*{Porque elegir usar Bitcoins en lugar de monedas clasicas}

\section*{Analisis BTC vs ETH vs BCH}

    \subsection*{Bitcoin}

    \subsubsection*{Precio con el que abre y precio con el que cierra}

    Haciendo primeramente un analisis visual, en general no se muestra una gran diferencia entre el precio con el que se abre y con el que se cierra el día con bitcoin. La diferencia de las medias es de solo 4 dolares que cierra mas el bitcoin en promedio durante toda su vida que con lo que abre.
    
    %[INSERTAR_GRAFICO_GLOBAL] 
    
    Entonces, quiza ver la diferencia por meses nos de una mejor idea de como los datos se puedan realacionar.

    Y ciertamente al menos ahora se puede ver que hay diferencias entre los precios en los que se abre y en los que se cierra.

    Después de ver por meses, se puede apreciar mejor las diferencias que hay: En general Los precios con los que abre la moneda se ven menores comparados con los que cierra. El mes en el que hubo menos cambio promedio entre la apertura y el cierre fue Septiembre de 2015 con 0.11 dolares y el de mayor cambio es Noviembre de 2017 con 100.92 dolares, aunque es posible que la media este inflada porque son solo 8 días a diferencia de todos los demás meses. Ignorando ese mes, el siguiente es Octubre de 2017
    con 68.7 dolares.

    %[INSERTAR_GRAFICOS_DIFERENCIAS]

    \subsubsection*{Precio más alto y más bajo de un día}

    Similar a lo ya calculado, el precio m\'as alto o m\'as bajo de un d\'ia dice mucho sobre como fluctua el precio.

    Visualmente con respecto a todos los a\~nos, se puede ver apenas las diferencias que hay entre el precio m\'as alto y m\'as bajo.

    %[INSERTAR_GRAFICO_ALTO_BAJO_GLOBAL]

    Entonces visualizar en unidades de tiempo m\'as peque\~nas podr\'ia mostrar m\'as de estos datos.

    La diferencia m\'axima que hubo entre el precio m\'as alto y bajo fue de 308.35 dolares en Septiembre de 2017.

    La diferencia m\'inima fue de 4.61 dolares en Agosto de 2013.

    Entonces Bitcoin en un solo d\'ia cambio en 300 dolares su precio.


    En los ultimos meses (octubre, septiembre, noviembre), que porcentaje del precio aumenta bitcoin desde que inicia el dia hasta que acaba?
    

    Por curiosidad, revisamos que tan seguido el precio máximo o mínimo era durante los periodos de apertura o cierre del día para las monedas monedas y todas se mantienen por debajo del 10\%.

    Para el Bitcoin, la más alta es el precio de apertura con el precio más bajo en el día, 7\% de las veces con 116 días y el más bajo fue cuando el precio con el que cierra es el más bajo, con solo 1.7\% que fueron 28 días.

    Para el Bitcoin, esto solo representa poco más del 15\% por lo que en su mayoría, los altos y los bajos de la moneda se daban no en el comienzo o final, si no durante el día.



    Similarmente para el Ethereum, la más alta es el precio de apertura con el precio más bajo en el día, 5.9\% de las veces con 49 días y el más bajo fue cuando el precio con el que cierra es el más bajo, con solo 2.7\% que fueron 22 días.

    Para el Ethereum, esto solo representa poco más del 18\% por lo que en su mayoría, los altos y los bajos de la moneda se daban no en el comienzo o final, si no durante el día.



    Finalmente para Bitcoin cash, el valor más alto es un 6.5\%, que para Bitcoin cash son 7 días, en el precio más bajo es el que cierra. El porcentaje más bajo es de 0.92\% ,o un día, en el que el precio más alto es con el que se cierra.
    
    Para Bitcoin Cash, esto solo representa menos del 18\% por lo que en su mayoría, los altos y los bajos de la moneda se daban no en el comienzo o final, si no durante el día.


    \subsubsection*{¿Es el número de Bitcoins representado por una función hipergeometrica?}

    En el árticulo original, Satoshi Nakamoto afirma que el número de Bitcoins está dado por una función Hipergeometrica y que gracias a esto, el número de Bitcoins que pueden ser minadas es finito.
    Veremos si esta afirmación es correcta o el autor mintió.

    Para ello veremos si el número de monedas en el tiempo se ajusta a una función hipergeometrica, y si es, con que parametros.

    %[INSERTAR_GRAFICO_NUM_BTC_EN_TIEMPO]

    %[HACER_PRUEBA_DE_AJUSTE]

    \subsubsection*{Cantidad de monedas y precio}

    Con un poco de conocimiento de economía, se puede relacionar la cantidad de monedas en un mercado con el precio que estas tienen. Entonces es de interes revisar si las criptomonedas cumplen con esto, es decir, si se puede ver dependencia entre la cantidad de criptomonedas y su precio.

    \subsubsection*{Blockchain}

    Como fue discutido al inicio, el protocolo Blockchain es lo que permitio que las criptomonedas existieran por haber solucionado el problema del gasto doble por lo que es de especial interes analizarlo al menos un poco, especificamente el tamaño del Blockchain ya que es uno de los datos que se tienen.

    Revisaremos si el tamaño del Blockchain y la cantidad de bitcoins son dependientes y si aumenta o disminuye con el tiempo.

    \subsubsection*{Hash rate}

    Hash rate (o taza de hasheo) se refiere a que tan poderosa puede ser una maquina de un minero de criptomonedas, especificamente se refiere al número de veces que una función hash es calculada por segundo. Las ganancias que un minero espera son directamente proporcionales al hash rate.

    El como se relaciona el hash rate con el tiempo puede ser de interes porque se podria ver como avanza el número de calculos que los mineros deben de realizar con las monedas a través del tiempo.

    %[INSERTAR_GRAFICO_HASH_RATE_TIEMPO]

    %[INSERTAR_CONCLUSIONES]

   Otro dato que se tiene del Bitcoin es la dificultad relativa de que tan difícil es encontrar un nuevo bloque. Debido a lo que es el hash rate, en primera instancia suena a que tiene que estar relacionado con la dificultad así que haremos una prueba de independencia entre estos datos. 

   \subsubsection*{Ingreso de los mineros}

    Actualmente a mucha gente le interesa minar Bitcoins debido a las ganancias que puede tener por lo que ver como ha cambiado el ingreso de los mineros puede ser interesante.

    %[INSERTAR_MINER_REVENUE_VS_TIEMPO]

    Individualmente a cada minero le interesa su ingreso así que ver de que puede depender es importante para estos. ¿Depende del número de monedas? Veamos:

    %[INSERTAR_MINER_REVENUE_VS_NUMERO_MONEDAS]

    %[HACER_PRUEBA_DEPENDENCIA]
    
    Entonces podriamos responder la pregunta, ¿Cuándo fue más rehabituable ser un minero?

    %[ANALIZAR_CUANDO_FUE_MAS_REHABITUABLE]







    La principal accion que realizaremos en el analisis es probar correlacion entre los datos y ver si es significativa la informacion que encontremos.
    \\
    Para ayudarnos, tambien interpretaremos graficos de los datos.
    \\
    Tambien intetaremos correlacionar Bitcoin con Ethereum.
    \\
    precio mas bajo en bitcoin cada anio
        se puede corresponder a algun suceso?
    Cambio mas abrupto de cresta a valle o viceversa del precio

    series de tiempo de las criptomonedas
        sobreponer la media de cada mes y anio
    sobreponer el precio mas alto y bajo de los dias
    Comparar bitcoin a las demas monedas
        mostrar las diferencias extremas entre bitcoin y las demas monedas
    hay dependencia entre el precio de las monedas?
        probar usando la media mensual
        probar usando la media global

\section*{Conclusiones}

	\subsection*{C\'omo es que el precio historico de las diferentes criptomonedas cambia en el tiempo?}
	\subsection*{Se puede predecir el precio de las criptomonedas?}
	\subsection*{Las criptomonedas son volatiles o estables?}
	\subsection*{Se relaciona la fluctiaci\'on de precio de una criptomoneda con otra?}
	\subsection*{Los cambios de precio se dan con respecto a temporadas?}
	\subsection*{predicciones de terceros de btc}
	\subsection*{predicciones propias}
	
\section{Bibliografia}





\end{document}
